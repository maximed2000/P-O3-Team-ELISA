\documentclass[landscape,12pt]{article}

\pagestyle{empty}  



\usepackage{pgfgantt}
\usepackage{ragged2e}
\def\pgfcalendarweekdayletter#1{%
	\ifcase#1M\or T\or W\or T\or F\or S\or S\fi%
}

\definecolor{foobarblue}{RGB}{0,153,255}
\definecolor{foobaryellow}{RGB}{234,187,0}
\definecolor{grey}{RGB}{170,170,170}

\newganttchartelement{foobar}{
	foobar/.style={
		shape= rectangle,
		inner sep=0pt,
		draw=grey!70!blue,
		thick,
		fill=white,
		inner color=blue!10, outer color=blue!40, opacity=0.95
	},
	foobar incomplete/.style={
		/pgfgantt/foobar,
		draw=foobaryellow,
		bottom color=foobaryellow!50
	},
	foobar label font=\slshape,
	foobar left shift=0,%.1,
	foobar right shift=0%-.1
}

\begin{document}
	
	% Een goede Gantt-chart maakt gebruik van \emph{links} en \emph{milestones}. Deze laatste definieer je in een lijstje onder de Gantt chart.
	
	\begin{figure}
	
		\begin{ganttchart}[hgrid, vgrid, x unit=5mm, time slot format=isodate,milestone label font=\tiny,group label font=\tiny,title label font=\tiny, foobar label font = \tiny]{2019-10-01}{2019-10-31}
			
			\gantttitlecalendar{week,month=shortname,} \\
			
			\ganttgroup{Inwerken}{2019-10-03}{2019-10-09} \\
			\ganttfoobar[name=29]{Taakverdeling}{2019-10-03}{2019-10-03}\\
			\ganttfoobar[name=29]{Overleg begeleiders}{2019-10-03}{2019-10-03}\\
			\ganttfoobar[name=29]{Brainstorm concepten}{2019-10-03}{2019-10-09}\\
			\ganttfoobar[name=29]{Literatuurstudie ELISA-test}{2019-10-03}{2019-10-06}\\
			\ganttgroup{Uitwerken concept}{2019-10-10}{2019-10-31} \\
			\ganttfoobar[name=29]{Selectie onderdelen}{2019-10-10}{2019-10-17} \\
			\ganttfoobar[name=29]{CAD-modellen maken}{2019-10-10}{2019-10-31} \\
			\ganttgroup{Testen onderdelen}{2019-10-17}{2019-10-31} \\
			\ganttfoobar[name=29]{Testen waterpomp}{2019-10-17}{2019-10-17} \\
			\ganttfoobar[name=29]{Testen steppermotor}{2019-10-24}{2019-10-31} \\
			\ganttgroup{Montage onderdelen}{2019-10-21}{2019-10-31} \\
			\ganttgroup{Programmeren}{2019-10-24}{2019-10-31} \\
			\ganttfoobar[name=29]{Steppermotor}{2019-10-24}{2019-10-31} \\
			\ganttgroup{Tussentijds verslag}{2019-10-17}{2019-10-31} \\
			
			
			
		\end{ganttchart}
	\end{figure}
	
	\clearpage	
	
	\begin{figure}
		
		\begin{ganttchart}[hgrid, vgrid, x unit=5mm, time slot format=isodate,milestone label font=\tiny,group label font=\tiny,title label font=\tiny, foobar label font = \tiny]{2019-11-01}{2019-11-30}
			\gantttitlecalendar{week,month=shortname} \\
			
			\ganttmilestone{Indienen tussentijds verslag}{2019-11-01} \\
			\ganttmilestone{Indienen \textit{peer assessment}}{2019-11-01} \\
			\ganttgroup{Montage onderdelen}{2019-11-1}{2019-11-07} \\
			\ganttgroup{Programmeren}{2019-11-1}{2019-11-30} \\
			\ganttfoobar[name=29]{Steppermotor}{2019-11-01}{2019-11-30} \\
			\ganttfoobar[name=29]{Waterpomp}{2019-11-1}{2019-11-30} \\
			
			
			
			
		\end{ganttchart}
	\end{figure}
	
	
	
	\begin{figure}
		\begin{ganttchart}[hgrid, vgrid, x unit=5mm, time slot format=isodate,milestone label font=\tiny,group label font=\tiny,title label font=\tiny, foobar label font = \tiny]{2019-12-1}{2019-12-20}
			\gantttitlecalendar{week,month=shortname} \\
			
			
			\ganttmilestone{Indienen eindverslag}{2019-12-17} \\
			\ganttmilestone{Presentatie \& demonstratie}{2019-12-19} \\
			\ganttmilestone{Indienen \textit{peer assessment}}{2019-12-20}
			
			
		\end{ganttchart}
	\end{figure}
	
	
	
	
\end{document}
